\documentclass[11pt,a4paper]{article}
\usepackage{algorithm}
\usepackage{algpseudocode}
\usepackage{mathtools}
\DeclarePairedDelimiter\ceil{\lceil}{\rceil}
\DeclarePairedDelimiter\floor{\lfloor}{\rfloor}
\usepackage[pdftex]{graphicx}
\usepackage{amsmath}
\begin{document}
\author{Ankur Dhoot}
\title{CS 580 HW 1}
\maketitle

\section*{Q1}

T(n) = $\theta$($n^{log_5 7}$lgn). Using the notation of the master method, a = 7, b = 5 so f(n) = $\theta$($n^{log_5 7}$) putting us in case 2 of the master method. Thus, $n^{log_5 7}$lgn is an asymptotic upper and lower bound on T(n).

\section*{Q2}
Claim: T(n) $\leq$ 45/8 cn

Proof: By induction on n

Base case: 

T(n) = c $\leq$ 45/8 cn for n $<$ 5

Induction Step: Assume the statement is true for n' $<$ n.

We must show the statement is true for n. 

T(n) = 
\begin{equation}
	\begin{aligned}
    T(\floor*{\frac{2n}{9}}) + 3T(\floor*{\frac{n}{5}}) + cn \\ 
    \leq a\floor*{\frac{2n}{9}} + 3a\floor*{\frac{n}{5}} + cn \\
    \leq a2n/9 + 3an/5 + cn \\
    = an(2/9 + 3/5) + cn \\
    = an(37/45) + cn = \\
    37/8cn + cn \\
    = 45/8cn \\
    = an
	\end{aligned}
\end{equation}

where the second step follows by the induction hypothesis. Thus, by the property of mathematical induction, the statement holds.

\section*{Q3}

\paragraph*{(a)}

Base case: a = 1
\begin{equation}
	\begin{aligned}
H(3)H(n-1) - H(1)H(n-2) - H(2)H(n-3) \\
= 2H(n-1) - H(n-3) \\
= H(n)
	\end{aligned}
\end{equation}
so the base case holds.

Induction step: Suppose true for a' $<$ a $<$ n - 2

We want to show the claim holds for a:
\begin{equation}
	\begin{aligned} 
	H(n) = H(a+2)H(n-a) - H(a)H(n-a-1) - H(a+1)H(n-a-2) \\ = [2H(a+1) - H(a-1)]H(n-a) - H(a)H(n-a-1) - H(a+1)H(n-a-2)
	\\ = H(a+1)[2H(n-a) - H(n-a-2)] - H(a-1)H(n-a) - H(a)H(n-a-1)
	\\ = H(a+1)H(n-a) - H(a-1)H(n-a) - H(a)H(n-a-1)
	\\ = H(n)
	\end{aligned}
\end{equation}
where the last step follows with the induction hypothesis applied with a - 1.
Thus, by the principle of mathematical induction the claim holds. 

\paragraph*{(b)}
Using a = k-1:
\begin{equation}
	\begin{aligned}
	H(2k) = H(k+1)H(2k-(k-1)) - H(k-1)H(2k-(k-1)-1) - H(k)H(2k-(k-1)-2) 
 \\= H(k+1)H(k+1) - H(k-1)H(k) - H(k)H(k-1) 
 \\= H(k+1)H(k+1) - 2H(k)H(k-1)
 	\end{aligned}
\end{equation}
\begin{equation}
	\begin{aligned}
	H(2k-1) = H(k+1)H(2k-1-(k-1) - H(k-1)H(2k-1-(k-1)-1) \\
	- H(k-1+1)H(2k-1-(k-1)-2)
	\\= H(k+1)H(k) - H(k-1)H(k-1) - H(k)H(k-2)
	\\= H(k+1)H(k) - H(k-1)H(k-1) - H(k)[2H(k) - H(k+1)]
	\\=2H(k+1)H(k) - 2H(k)H(k) - H(k-1)H(k-1)
 	\end{aligned}
\end{equation}

\paragraph*{(c)}
Using a = k-1:
\begin{equation}
	\begin{aligned}
	H(2k+1) = H(k-1+2)H(2k+1-(k-1)) - H(k-1)H(2k+1-(k-1)-1) 
	\\- H(k-1+1)H(2k+1-(k-1)-2) 
	\\= H(k+1)H(k+2) - H(k-1)H(k+1) - H(k)H(k) 
	\\= H(k+1)[2H(k+1) - H(k-1)] - H(k-1)H(k+1) - H(k)H(k)
	\\= 2H(k+1)H(k+1) - 2H(k+1)H(k-1) - H(k)H(k)
 	\end{aligned}
\end{equation}

\begin{equation}
	\begin{aligned}
	H(2k-2) = H(k-1+2)H(2k-2-(k-1)) - H(k-1)H(2k-2-(k-1)-1) 
	\\- H(k-1+1)H(2k-2-(k-1)-2)
	\\= H(k+1)H(k-1) - H(k-1)H(k-2) - H(k)H(k-3)
	\\= H(k+1)H(k-1) -H(k-1)[2H(k) - H(k+1)] - H(k)[2H(k-1) - H(k)]
	\\=2H(k+1)H(k-1) - 4H(k)H(k-1) + H(k)H(k)
 	\end{aligned}
\end{equation}

\paragraph*{(d)}
\end{document}