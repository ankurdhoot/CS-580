\documentclass[11pt,a4paper]{article}
\usepackage{graphicx}
\usepackage{titling}
\usepackage{pdfpages}
\usepackage{tikz}
\usetikzlibrary{shapes}
\usepackage{amssymb}
\usepackage{forest}
\begin{document}
\author{Ankur Dhoot}
\title{CS 580 HW 5}
\maketitle

\section*{Q1}

\paragraph*{(a)}

We'll give an O($E^{2}$) time algorithm for constructing a valid overflow in G. We'll iteratively find an edge which doesn't satisfy the lower bound (i.e f(u,v) $<$ l(u,v)). Let $c_{P}$ = l(u, v) - f(u,v) for the edge u-v not satisfying the lower bound. Then, we'll find a path from s to u and v to t in G. This can be done in O(E) time using BFS or DFS to find the s-u path and then the v-t path. Then, for all edges on that path, we'll increase the flow value of each edge by $c_{P}$. After we've increased the flow, edge (u,v) now satisfies the lower bound as do all edges that previously had satisfied the lower bound. 

Each iteration runs in O(E) time. Since there are E edges, and at least one more edge satisfies the lower bound after each iteration, we iterate at most O(E) times. Thus, the total running time is O($E^{2}$).

\paragraph*{(b)}
We'll reduce the minimum flow problem given an overflow to a maximum flow problem.

Let f be a valid overflow in G. Consider $G_{f}'$. If (u,v) $\in$ E, let c(u,v) = $\infty$ (since edges don't have restrictions on the maximum amount of flow that can be pushed through them). If (u,v) $\in$ E and f(u,v) > l(u,v), let c(v, u) = f(u,v) - l(u,v) (i.e the maximum amount possible by which we can reduce f(u,v) while maintaining a valid overflow). Let s' = t, and t' = s. Let f' be the maximum s'-t' flow in $G_{f}'$. It's clear that if we update our original overflow using this max flow by letting $f^{*}$(u,v) = f(u,v) + f'(u,v) - f'(v,u), the the flow $f^{*}$ is a minimum flow in G. 

Since we can construct $G_{f}'$ in O(V + E) time and we can find the maximum flow $G_{f}'$ in O($VE^{2}$) time using Edmonds-Karp, we can find the minimum flow in O($VE^{2}$) by the above method.

\paragraph*{(c)}
(1) $\Rightarrow$ (2): Suppose for the sake of contradiction the f is a minimum flow in G but that $G_{f}'$ contains a path from s' to t', say P. Consider the capacities added to $G_{f}'$ as in (b). Then, there exists some minimum capacity edge along P. If we augment the flow from s' to t' along path P by this minimum capacity, the resulting flow is a valid overflow and of lesser flow value, a contradiction. Thus, there can exist no path from s' to t' in $G_{f}'$. 

(2) $\Rightarrow$ (3): Suppose that $G_{f}'$ contains no path from s' to t'. Let T = \{v $\in$ V: there exists a path from s' to v in $G_{f}'$\}. Let S = V - T. The partition (S, T) is a cut since s'=t $\in$ T and t'=s $\in$ S because there is no s' to t' path in $G_{f}'$. Now consider a pair of vertices u $\in$ S and v $\in$ T. If (v,u) $\in$ E, then (v,u) $\in$ $E_{f}'$. But then u $\in$ T, a contradiction. Thus (u,v) $\not\in$ E. If (u,v) $in$ E, we must have f(u,v) = l(u,v) otherwise f(u,v) > l(u,v) meaning (v,u) $in$ $E_{f}'$ which would place u $\in$ T, a contradiction. 

Thus, the cut above has no edges from T to S in G, and for any edge (u,v), we have f(u,v) = l(u,v). Thus, $|f|$ = l(S,T) for some (S,T) cut having no edges from T to S.

(3) $\Rightarrow$ (1): Suppose $|f|$ = l(S,T) for some (S,T) cut having no edges from T to S. 

Claim: $|f|$ $\geq$ l(S,T) for any (S,T) cut having no edges from T to S. 

Proof: $|f|$ = f(S,T) 

= $\sum\limits_{u \in S}\sum\limits_{v \in T} f(u,v)$ - $\sum\limits_{u \in S}\sum\limits_{v \in T} f(v,u)$

= $\sum\limits_{u \in S}\sum\limits_{v \in T} f(u,v)$

$\geq$ $\sum\limits_{u \in S}\sum\limits_{v \in T} l(u,v)$

= l(S,T)


Thus, since $|f|$ $\geq$ l(S,T) for any (S,T) cut having no edges from T to S, and by hypothesis we have that $|f|$ = l(S,T) for some (S,T) cut having no edges from T to S $\Rightarrow$ f is a minimum flow.

\end{document}